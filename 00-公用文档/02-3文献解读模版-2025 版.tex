\documentclass[10pt,aspectratio=169,compress,ignorenonframetext]{beamer}

\usepackage[UTF8,noindent]{ctex}
\newcommand{\filter}[1]{}
\usefonttheme{serif}              % 使用衬线字体
\usefonttheme{professionalfonts}  % 数学公式字体
\usepackage{mathtools}
\usepackage{hyperref}
\usepackage{booktabs}
\usepackage{verbatim}
\usepackage{anyfontsize}
\usepackage{setspace}
\usepackage{multirow}
\usepackage[table]{xcolor}

\usepackage[backend=biber,style=chinese-erj,uniquename=false]{biblatex}
\renewcommand{\parencite}{\cite} % 如果 biblatex-gb7714-2015 较新可删除
%\addbibresource{macro.bib}

\renewcommand{\baselinestretch}{1.3}

\usepackage{fontspec}
\mode<presentation>
{
\usetheme{Singapore}
\setbeamercovered{transparent}
}
\usepackage{color}

\AtBeginSection[]{
}

\usepackage{graphicx} % 用于插入图片
\usepackage{subcaption} % 用于 \subcaption 环境

\makeatletter
\setbeamertemplate{footline}
{
    \leavevmode%
    \hbox{%
        \begin{beamercolorbox}[wd=.333333\paperwidth,ht=2.25ex,dp=1ex,center]{author in head/foot}%
            \usebeamerfont{author in head/foot}\insertshortauthor
        \end{beamercolorbox}%
        \begin{beamercolorbox}[wd=.333333\paperwidth,ht=2.25ex,dp=1ex,center]{title in head/foot}%
            \usebeamerfont{title in head/foot}\insertshorttitle
        \end{beamercolorbox}%
        \begin{beamercolorbox}[wd=.333333\paperwidth,ht=2.25ex,dp=1ex,right]{date in head/foot}%
            \usebeamerfont{date in head/foot}\insertshortdate{}\hspace*{2em}
            \insertframenumber{} / \inserttotalframenumber\hspace*{2ex} 
        \end{beamercolorbox}
     }%
     \vskip 0pt%
}

%% --> 导言页
\title[论文名称]{\Huge 拟解读的论文名称,需要保留作者名、期刊名、年(这些必要信息可以换行)}

\author{解读人: XXX}
\institute{武汉大学金融系}

\date{展示日期}
 
\begin{document}

\begin{frame}
\titlepage
\end{frame}


\begin{frame}{一页总结}

建议在标题页时,通过提问的方式启发思考:给定题目,大家会怎么做这个研究?具体来说,给定题目,大家会提出什么样的研究问题、相应的研究假设、研究设计等。

这一页用于总结(可绘图)。
\end{frame}



\section{研究问题}

\begin{frame}{研究问题}

这一模块希望达到的学习目标是(ILO1):How to critically evaluate the quality of a research idea


每个部分可以多页,但切勿过多。

第一部分:What are the research questions (summarize in one or two sentences)? 如有必要,可以给出子问题,但不宜过多。

第二部分:Why are the research questions interesting? 这里非常重要,请用条目式列出。切记不能仅用没有人研究过认为值得研究。

第三部分:What is the paper's contribution? 
\begin{enumerate}
	\item Find the literature
	\item Summarize the literature(这一部分一定要有)
	\item Summarize the marginal contributions to the literature
\end{enumerate}

\end{frame}

\section{研究假设}

\begin{frame}{研究假设}

这一部分的学习目标(ILO2): How to develop a testable research hypothesis

第四部分:What hypotheses are tested in the paper? list them explicitly. 用条目式明确地列出。

\begin{enumerate}
	\item Do these hypotheses follow from and answer the research questions?
	\item Do these hypotheses follow from theory or are they otherwise adequately developed? Please explain the logic of the hypotheses (use visualization if possible).
\end{enumerate}

\end{frame}

\section{研究设计}

\begin{frame}{研究设计}

本部分的学习目标(ILO3):How to create a sound research design to test a research hypothesis。

需要分析如下的部分(可分页)

第五部分:Sample: comment on the appropriateness of the sample selection procedures.

第六部分:Dependent and independent variables: comment on the appropriateness of variable definition and measurement (focus on the key dependent variables and independent variables).

第七部分:Regression/prediction model specification: comment on the appropriateness of the regression/prediction model specification.

\end{frame}

\section{研究结果}

\begin{frame}{研究结果}

本部分的学习目标(ILO4): How to interpret and communicate the results of the research design

第八部分:What difficulties arise in drawing inferences from the empirical work?


针对研究假设,一个个地分析研究结果,是否能够检验研究假设?几个研究假设,几页(如果论文用多个结果回答一个目标,列出来)

 如这一页,需要列出:研究假设1、模型设定、研究结果(表格或图,并聚焦于研究假设,给出结论:是否检验了没有?有何难点?)
 
 再下一页,列出:研究假设2、模型设定、研究结果(表格或图,并聚焦于研究假设,给出结论:是否检验了没有?有何难点?)

依次类推。


\end{frame}

\section{我的思考}

\begin{frame}{思考}

最后一部分需要给出自己对该论文的思考。这一部分可以在大家讨论之后再给出。

第九部分:Describe at least one publishable and feasible extension of this research. 条目式列出,简要解释。如有可能,列出可能的边际贡献。如有可能,有初步结果更好。

\end{frame}

\section*{附录}

\begin{frame}{附录}

所有其他的东西(未展示的论文结果、复现结果等,参考文献,相关的论文,如后续接着展开的论文等),放在附录中。附录可不讲,仅用于备用,不限页数。

\end{frame}

\end{document} 